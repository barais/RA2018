We have selected three main contributions for \team's research axis \#4:
one is in the field of runtime management of resources for dynamically adaptive system, one in the field of temporal context model for dynamically adaptive system and a last one 
to improve the exploration of hidden real-time structures of programming behavior at runtime. 


\paragraph{Resource-aware models@runtime layer for dynamically adaptive system}

In Kevoree, one of the goal is to work on the shipping pases in which we aim at making deployment, and the reconfiguration simple and accessible to a whole development team. This year, we mainly explore two main axes.

In the first one, we try to improve the proposed models that could be used at runtime to improve resource usage in two domains: cloud computing and energy~\cite{dartois:hal-01898438}. In the cloud computing domain, we try to improve resources usage in providing models to cloud provider to allow the reselling of unused resources to peers. Indeed, although Cloud computing techniques have reduced the total cost of ownership thanks to virtualization, the average usage of resources (e.g., CPU, RAM, Network, I/O) remains low. To address such issue, one may sell unused resources. Such a solution requires the Cloud provider to determine the resources available and estimate their future use to provide availability guarantees. In this work,  we propose a technique that uses machine learning algorithms (Random Forest, Gradient Boosting Decision Tree, and Long Short Term Memory) to forecast 24-hour of available resources at the host level. Our technique relies on the use of quantile regression to provide a flexible trade-off between the potential amount of resources to reclaim and the risk of SLA violations. In addition, several metrics (e.g., CPU, RAM, disk, network) were predicted to provide exhaustive availability guarantees. Our methodology was evaluated by relying on four in production data center traces and our results show that quantile regression is relevant to reclaim unused resources. Our approach may increase the amount of savings up to 20\% compared to traditional approaches.\\

In the energy domain, we work at proposing models that could be used at runtime to improve self-consumption of renewable energies~\cite{rio:hal-01913169}. Self-consumption of renewable energies is defined as electricity that is produced from renewable energy sources, not injected to the distribution or transmission grid or instantaneously withdrawn from the grid and consumed by the owner of the power production unit or by associates directly contracted to the producer. Designing solutions in favor of self-consumption for small industries or city districts is challenging. It consists in designing an energy production system made of solar panels, wind turbines, batteries that fit the annual weather prediction and the industrial or human activity. In this context, this we highlight the essentials of a domain specific modeling language designed to let domain experts run their own simulations. 


\paragraph{A Temporal Model for Interactive Diagnosis of Adaptive Systems}



The evolving complexity of adaptive systems impairs our ability to deliver anomaly-free solutions. Fixing these systems
require a deep understanding on the reasons behind decisions which led to faulty or suboptimal system states. Developers thus
need diagnosis support that trace system states to the previous circumstances targeted requirements, input context that had
resulted in these decisions. However, the lack of efficient temporal representation limits the tracing ability of current approaches. To tackle this problem, we describe a novel temporal data model to represent, store and query decisions as well as their relationship with the knowledge (context, requirements, and actions)~\cite{mouline:hal-01862964}. We validate our approach through a use case based-on the smart grid at Luxembourg.

Based on this work, we also enable a models@runtime approach in which we integrate the time required for a reconfiguration action to achieve the expected impact~\cite{mouline:hal-01723451}.
Indeed in most of the MAPE-K loop system, unfinished actions as well as their expected effects over time are not taken into consideration in MAPE-K loop processes, leading upcoming analysis phases potentially take sub-optimal actions. In this work, we propose an extended context model for MAPE-K loop that integrates the history of planned actions as well as their expected effects over time into the context representations. This information can then be used during the upcoming analysis and planning phases to compare measured and expected context metrics. We demonstrate on a cloud elasticity manager case study that such temporal action-aware context leads to improved reasoners while still be highly scalable.



\paragraph{Detection and analysis of behavioral T-patterns in debugging activities}
A growing body of research in empirical software engineering applies recurrent patterns analysis in order to make sense  of the developers’ behavior during their interactions with IDEs. However, the exploration of hidden real-time structures of programming behavior remains a challenging task. In this work~\cite{sotovalero:hal-01763369}, we investigate the presence of temporal behavioral patterns (T-patterns) in debugging activities using the THEME software. Our preliminary exploratory results show that debugging activities are strongly correlated with code editing, file handling, window interactions and other general types of programming activities. The validation of our T-patterns detection approach demonstrates that debugging activities are performed on the basis of repetitive and well-organized behavioral events. Furthermore, we identify a large set of T-patterns that associate debugging activities with build success, which corroborates the positive impact of debugging practices on software development.






