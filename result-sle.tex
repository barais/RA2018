\paragraph{On Language Interfaces}
Complex systems are developed by teams of experts from multiple domains , who can be liberated from becoming programming experts through domain-specific languages (DSLs). The implementation of the different concerns of DSLs (including syntaxes and semantics) is now well-established and supported by various languages workbenches. However, the various services associated to a DSL (e.g., editors, model checker, debugger or composition operators) are still directly based on its implementation. Moreover, while most of the services crosscut the different DSL concerns, they only require specific information on each. Consequently, this prevents the reuse of services among related DSLs, and increases the complexity of service implementation. Leveraging the time-honored concept of interface in software engineering, we discuss in \cite{degueule:hal-01424909} the benefits of language interfaces in the context of software language engineering. In particular, we elaborate on particular usages that address current challenges in language development.

\paragraph{Safe Model Polymorphism for Flexible Modeling}
Domain-Specific Languages (DSLs) are increasingly used by domain experts to handle various concerns in systems and software development. To support this trend, the Model-Driven Engineering (MDE) community has developed advanced techniques for designing new DSLs. However, the widespread use of independently developed, and constantly evolving DSLs is hampered by the rigidity imposed to the language users by the DSLs and their tooling, e.g., for manipulating a model through various similar DSLs or successive versions of a given DSL. In \cite{degueule:hal-01367305}, we propose a disciplined approach that leverages type groups' polymorphism to provide an advanced type system for manipulating models, in a polymorphic way, through different DSL interfaces. A DSL interface, aka. model type, specifies a set of features, or services, available on the model it types, and subtyping relations among these model types define the safe substitutions. This type system complements the Melange language workbench and is seamlessly integrated into the Eclipse Modeling Framework (EMF), hence providing structural interoperability and compatibility of models between EMF-based tools. We illustrate the validity and practicability of our approach by bridging safe interoperability between different semantic and syntactic variation points of a finite-state machine (FSM) language, as well as between successive versions of the Unified Modeling Language (UML).

\paragraph{Revisiting Visitors for Modular Extension of Executable DSMLs}
Executable Domain-Specific Modeling Languages (xDSMLs) are typically defined by metamodels that specify their abstract syntax, and model interpreters or compilers that define their execution semantics. To face the proliferation of xDSMLs in many domains, it is important to provide language engineering facilities for opportunistic reuse, extension, and customization of existing xDSMLs to ease the definition of new ones. Current approaches to language reuse either require to anticipate reuse, make use of advanced features that are not widely available in programming languages, or are not directly applicable to metamodel-based xDSMLs. In \cite{leduc:hal-01568169}, we propose a new language implementation pattern, named REVISITOR, that enables independent extensibility of the syntax and semantics of metamodel-based xDSMLs with incremental compilation and without anticipation. We seamlessly implement our approach alongside the compilation chain of the Eclipse Modeling Framework, thereby demonstrating that it is directly and broadly applicable in various modeling environments. We show how it can be employed to incrementally extend both the syntax and semantics of the fUML language without requiring anticipation or re-compilation of existing code, and with acceptable performance penalty compared to classical handmade visitors.

\paragraph{Advanced and efficient execution trace management for executable domain-specific modeling languages}
Executable Domain-Specific Modeling Languages (xDSMLs) enable the application of early dynamic verification and validation (V\&V) techniques for behavioral models. At the core of such techniques, execution traces are used to represent the evolution of models during their execution. In order to construct execution traces for any xDSML, generic trace metamodels can be used. Yet, regarding trace manipulations, generic trace metamodels lack efficiency in time because of their sequential structure, efficiency in memory because they capture superfluous data, and usability because of their conceptual gap with the considered xDSML. We contributed in \cite{bousse:hal-01614377} a novel generative approach that defines a multidimensional and domain-specific trace metamodel enabling the construction and manipulation of execution traces for models conforming to a given xDSML. Efficiency in time is improved by providing a variety of navigation paths within traces, while usability and memory are improved by narrowing the scope of trace metamodels to fit the considered xDSML. We evaluated our approach by generating a trace metamodel for fUML and using it for semantic differencing, which is an important V\&V technique in the realm of model evolution. Results show a significant performance improvement and simplification of the semantic differencing rules as compared to the usage of a generic trace metamodel.

\paragraph{Omniscient Debugging for Executable DSLs}

Omniscient debugging is a promising technique that relies on execution traces to enable free traversal of the states reached by a model (or program) during an execution. While a few General-Purpose Languages (GPLs) already have support for omniscient debugging, developing such a complex tool for any executable Domain Specific Language (DSL) remains a challenging and error prone task. A generic solution must: support a wide range of executable DSLs independently of the metaprogramming approaches used for implementing their semantics; be efficient for good responsiveness. Our contribution in \cite{bousse:hal-01662336} relies on a generic omniscient debugger supported by efficient generic trace management facilities. To support a wide range of executable DSLs, the debugger provides a common set of debugging facilities, and is based on a pattern to define runtime services independently of metaprogramming approaches. Results show that our debugger can be used with various executable DSLs implemented with different metaprogramming approaches. As compared to a solution that copies the model at each step, it is on average six times more efficient in memory, and at least 2.2 faster when exploring past execution states, while only slowing down the execution 1.6 times on average.

\paragraph{Reverse Engineering Language Product Lines from Existing DSL Variants}

The use of domain-specific languages (DSLs) has become a successful technique in the development of complex systems. Nevertheless, the construction of this type of languages is time-consuming and requires highly-specialized knowledge and skills. An emerging practice to facilitate this task is to enable reuse through the definition of language modules which can be later put together to build up new DSLs. 
In \cite{mendezacuna:hal-01524632}, we propose a reverse-engineering technique to ease-off such a development scenario. Our approach receives a set of DSL variants which are used to automatically recover a language modular design and to synthesize the corresponding variability models. The validation is performed in a project involving industrial partners that required three different variants of a DSL for finite state machines. This validation shows that our approach is able to correctly identify commonalities and variability.

\paragraph{Software Language Engineering for Virtual Reality Software Development}

Due to the nature of Virtual Reality (VR) research, conducting experiments in order to validate the researcher's hypotheses is a must. 
However, the development of such experiments is a tedious and time-consuming task. 
In \cite{lemoulec:hal-01549042}, we propose to make this task easier, more intuitive and faster with a method able to describe and generate the most tedious components of VR experiments. 
The main objective is to let experiment designers focus on their core tasks: designing , conducting, and reporting experiments. 
To that end, we applied well-established SLE concepts promoted in \team{} to the VR domain to ease the development of VR experiments.
More precisely, we propose the use of DSLs to ease the description and generation of VR experiments. 
An analysis of published VR experiments is used to identify the main properties that characterize VR experiments. 
This allowed us to design AGENT (Automatic Generation of ExperimeNtal proTocol runtime), a DSL for specifying and generating experimental protocol runtimes. 
We demonstrated the feasibility of our approach by using AGENT on two experiments published in the VRST'16 proceedings. 

\paragraph{Create and Play your Pac-Man Game with the GEMOC Studio}

Executable Domain-Specific Languages (DSLs) are used for defining the behaviors of systems. In particular, the
operational semantics of such DSLs may define how conforming
models react to stimuli from their environment. This commonly
requires adapting the semantics to define both the possible
domain-level stimuli, and their handling during the execution.
However, manually adapting the semantics for such cross-cutting
concern is a complex and error-prone task. In \cite{}, we
demonstrate a tool addressing this problem by allowing the
augmentation of operational semantics for handling stimuli, and
by automatically generating a complete behavioral language
interface from this augmentation. At runtime, this interface can receive stimuli sent to models, and can safely handle them by automatically interrupting the execution flow. This tool is an extension to the GEMOC Studio, a language and modeling
workbench for executable DSLs We demonstrate how it can be
used to implement a Pac-Man DSL enabling to create and play
Pac-Man games.
