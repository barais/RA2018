\paragraph{Omniscient Debugging for Executable DSLs}

Omniscient debugging is a promising technique that relies on execution traces to enable free traversal of the states reached by a model (or program) during an execution. While a few General-Purpose Languages (GPLs) already have support for omniscient debugging, developing such a complex tool for any executable Domain Specific Language (DSL) remains a challenging and error prone task. A generic solution must: support a wide range of executable DSLs independently of the metaprogramming approaches used for implementing their semantics; be efficient for good responsiveness. Our contribution in \cite{bousse:hal-01662336} relies on a generic omniscient debugger supported by efficient generic trace management facilities. To support a wide range of executable DSLs, the debugger provides a common set of debugging facilities, and is based on a pattern to define runtime services independently of metaprogramming approaches. Results show that our debugger can be used with various executable DSLs implemented with different metaprogramming approaches. As compared to a solution that copies the model at each step, it is on average six times more efficient in memory, and at least 2.2 faster when exploring past execution states, while only slowing down the execution 1.6 times on average.

\paragraph{Trace Comprehension Operators for Executable DSLs}

Recent approaches contribute facilities to breathe life into metamodels, thus making behavioral models directly executable. Such facilities are particularly helpful to better utilize a model over the time dimension, e.g., for early validation and verification. However, when even a small change is made to the model, to the language definition (e.g., semantic variation points), or to the external stimuli of an execution scenario, it remains difficult for a designer to grasp the impact of such a change on the resulting execution trace. This prevents accessible trade-off analysis and design-space exploration on behavioral models. In \cite{leroy:hal-01803031}, we propose a set of formally defined operators for analyzing execution traces. The operators include dynamic trace filtering, trace comparison with diff computation and visualization, and graph-based view extraction to analyze cycles. The operators are applied and validated on a demonstrative example that highlight their usefulness for the comprehension specific aspects of the underlying traces.

\paragraph{Model Transformation Reuse across Metamodels}
Model transformations (MTs) are essential elements of model-driven engineering (MDE) solutions. MDE promotes the creation of domain-specific metamodels, but without proper reuse mechanisms, MTs need to be developed from scratch for each new metamodel. In \cite{bruel:hal-01910113}, awarded by the \textbf{best paper award at ICMT 2018}, we classify reuse approaches for MTs across different metamodels and compare a sample of specific approaches -- model types, concepts, a-posteriori typing, multilevel modeling, and design patterns for MTs -- with the help of a feature model developed for this purpose, as well as a common example. We discuss strengths and weaknesses of each approach, provide a reading grid used to compare their features, and identify gaps in current reuse approaches.

\paragraph{Modular Language Composition for the Masses}
The goal of modular language development is to enable the definition of new languages as assemblies of pre-existing ones. Recent approaches in this area are plentiful but usually suffer from two main problems: either they do not support modular language composition both at the specification and implementation levels, or they require advanced knowledge of specific paradigms which hampers wide adoption in the industry. In \cite{leduc:hal-01890446}, awarded by the \textbf{best artefact award at SLE 2018}, we introduce a non-intrusive approach to modular development of language concerns with well-defined interfaces that can be composed modularly at the specification and implementation levels. We present an implementation of our approach atop the Eclipse Modeling Framework, namely Alex-an object-oriented metalanguage for semantics definition and language composition. We evaluate Alex in the development of a new DSL for IoT systems modeling resulting from the composition of three independently defined languages (UML activity diagrams, Lua, and the OMG Interface Description Language). We evaluate the effort required to implement and compose these languages using Alex with regards to similar approaches of the literature.

\paragraph{Shape-Diverse DSLs}
Domain-Specific Languages (DSLs) manifest themselves in remarkably diverse shapes, ranging from internal DSLs embedded as a mere fluent API within a programming language, to external DSLs with dedicated syntax and tool support. Although different shapes have different pros and cons, combining them for a single language is problematic: language designers usually commit to a particular shape early in the design process, and it is hard to reconsider this choice later. In the new ideas paper \cite{coulon:hal-01889155} awarded as the \textbf{best new ideas paper at SLE 2018}, we envision a language engineering approach enabling (i) language users to manipulate language constructs in the most appropriate shape according to the task at hand, and (ii) language designers to combine the strengths of different technologies for a single DSL. We report on early experiments and lessons learned building Prism, our prototype approach to this problem. We illustrate its applicability in the engineering of a simple shape-diverse DSL implemented conjointly in Rascal, EMF, and Java. We hope that our initial contribution will raise the awareness of the community and encourage future research.

\paragraph{Fostering metamodels and grammars}
Advanced and mature language workbenches have been proposed in the past decades to develop Domain-Specific Languages (DSL) and rich associated environments. They all come in various flavors, mostly depending on the underlying technological space (e.g., grammarware or modelware). However, when the time comes to start a new DSL project, it often comes with the choice of a unique technological space which later bounds the possible expected features. In \cite{lelandais:hal-01910139}, we introduce NabLab, a full-fledged industrial environment for scientific computing and High Performance Computing (HPC), involving several metamodels and grammars. Beyond the description of an industrial experience of the development and use of tool-supported DSLs, we report in this paper our lessons learned, and demonstrate the benefits from usefully combining metamodels and grammars in an integrated environment.

\paragraph{Software Language Engineering for Virtual Reality Software Development}

Due to the nature of Virtual Reality (VR) research, conducting experiments in order to validate the researcher's hypotheses is a must. 
However, the development of such experiments is a tedious and time-consuming task. 
In \cite{lemoulec:hal-01549042}, we propose to make this task easier, more intuitive and faster with a method able to describe and generate the most tedious components of VR experiments. 
The main objective is to let experiment designers focus on their core tasks: designing , conducting, and reporting experiments. 
To that end, we applied well-established SLE concepts promoted in \team{} to the VR domain to ease the development of VR experiments.
More precisely, we propose the use of DSLs to ease the description and generation of VR experiments. 
An analysis of published VR experiments is used to identify the main properties that characterize VR experiments. 
This allowed us to design AGENT (Automatic Generation of ExperimeNtal proTocol runtime), a DSL for specifying and generating experimental protocol runtimes. 
We demonstrated the feasibility of our approach by using AGENT on two experiments published in the VRST'16 proceedings. 

\paragraph{A Unifying Framework for Homogeneous Model Composition}

The growing use of models for separating concerns in complex systems has lead to a proliferation of model composition operators. These composition operators have traditionally been defined from scratch following various approaches differing in formality, level of detail, chosen paradigm, and styles. Due to the lack of proper foundations for defining model composition (concepts, abstractions, or frameworks), it is difficult to compare or reuse composition operators. In \cite{kienzle:hal-01949050}, we stipulate the existence of a unifying framework that reduces all structural composition operators to structural merging, and all composition operators acting on discrete behaviors to event scheduling. We provide convincing evidence of this hypothesis by discussing how structural and behavioral homogeneous model composition operators (i.e., weavers) can be mapped onto this framework. Based on this discussion, we propose a conceptual model of the framework, and identify a set of research challenges, which, if addressed, lead to the realization of this framework to support rigorous and efficient engineering of model composition operators for homogeneous and eventually heterogeneous modeling languages.

\paragraph{Concern-Oriented Language Development (COLD)}

Domain-Specific Languages (DSLs) bridge the gap between the problem space, in which stakeholders work, and the solution space, i.e., the concrete artifacts defining the target system. They are usually small and intuitive languages whose concepts and expressive-ness fit a particular domain. DSLs recently found their application in an increasingly broad range of domains, e.g., cyber-physical systems, computational sciences and high performance computing. Despite recent advances, the development of DSLs is error-prone and requires substantial engineering efforts. Techniques to reuse from one DSL to another and to support customization to meet new requirements are thus particularly welcomed. Over the last decade, the Software Language Engineering (SLE) community has proposed various reuse techniques. However, all these techniques remain disparate and complicate the development of real-world DSLs involving different reuse scenarios. In \cite{combemale:hal-01803008}, we introduce the Concern-Oriented Language Development (COLD) approach, a new language development model that promotes modularity and reusability of language concerns. A language concern is a reusable piece of language that consists of usual language artifacts (e.g., abstract syntax, concrete syntax, semantics) and exhibits three specific interfaces that support (1) variability management, (2) customization to a specific context, and (3) proper usage of the reused artifact. The approach is supported by a conceptual model which introduces the required concepts to implement COLD. We also present concrete examples of some language concerns and the current state of their realization with metamodel-based and grammar-based language workbenches. We expect this work to provide insights into how to foster reuse in language specification and implementation, and how to support it in language workbenches.

\paragraph{Tool-Support of Socio-Technical Coordination in the Context of Heterogeneous Modeling}

The growing complexity of everyday life systems (and devices) over the last decades has forced the industry to use and investigate different development techniques to manage the many different aspects of the systems. In this context, the use of model-driven engineering (MDE) has emerged and is now common practice for many engineering disciplines. However, this comes with important challenges. As a set of main challenges relates to the fact that different modeling techniques, languages, and tools are required to deal with the different system aspects, and that support is required to ensure consistency and coherence between the different models. In \cite{bordeleau:hal-01958443}, we identify a number of challenges and propose a roadmap on how tooling can support a multi-model integrated way of working.